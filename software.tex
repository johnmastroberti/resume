% Preamble {{{
%% start of file `template.tex'.
%% Copyright 2006-2013 Xavier Danaux (xdanaux@gmail.com).
%
% This work may be distributed and/or modified under the
% conditions of the LaTeX Project Public License version 1.3c,
% available at http://www.latex-project.org/lppl/.


\documentclass[11pt,a4paper,sans]{moderncv}        % possible options include font size ('10pt', '11pt' and '12pt'), paper size ('a4paper', 'letterpaper', 'a5paper', 'legalpaper', 'executivepaper' and 'landscape') and font family ('sans' and 'roman')

% moderncv themes
\moderncvstyle{classic}                             % style options are 'casual' (default), 'classic', 'oldstyle' and 'banking'
\moderncvcolor{red}                               % color options 'blue' (default), 'orange', 'green', 'red', 'purple', 'grey' and 'black'
%\renewcommand{\familydefault}{\sfdefault}         % to set the default font; use '\sfdefault' for the default sans serif font, '\rmdefault' for the default roman one, or any tex font name
%\nopagenumbers{}                                  % uncomment to suppress automatic page numbering for CVs longer than one page

% character encoding
\usepackage[utf8]{inputenc}                       % if you are not using xelatex ou lualatex, replace by the encoding you are using
%\usepackage{CJKutf8}                              % if you need to use CJK to typeset your resume in Chinese, Japanese or Korean

% adjust the page margins
\usepackage[scale=0.75]{geometry}
%\setlength{\hintscolumnwidth}{3cm}                % if you want to change the width of the column with the dates
%\setlength{\makecvtitlenamewidth}{10cm}           % for the 'classic' style, if you want to force the width allocated to your name and avoid line breaks. be careful though, the length is normally calculated to avoid any overlap with your personal info; use this at your own typographical risks...

\newcommand{\techlist}[1]{\textbf{Key technologies: {#1}}\\}
%}}}

% personal data
\name{John}{Mastroberti}
\title{Resumé}
\address{1426 State Road 135 S}{Nashville, IN 47448}
\phone[mobile]{+1~(843)~907~3619}
\email{johnmastroberti123@gmail.com}
\homepage{johnmm.xyz}
\photo[64pt][0pt]{picture}
\begin{document}
\makecvtitle

\section{Education}
% Three items {{{
\cventry{2016--2019}{B.A., Cornell University}
{College of Arts and Sciences}{Ithaca, NY}
{\textit{GPA: 3.932}}
{Majors: Physics (Magna Cum Laude), Math (Cum Laude)}

\cventry{2020--2021}{M.S., Indiana University}
{Department of Physics}{Bloomington, IN}{\textit{GPA: 4.0}}{}

\cventry{2021--present}{Ph.D. student, Indiana University}
{Department of Physics}{Bloomington, IN}{}
{Concentration: experimental neutrino physics}
%}}}


\section{Work Experience}
\subsection{Software Development}
% COHERENT
\cventry{2020--present}{Neutrino Detector Simulation}%{{{
{Rex Tayloe, Dan Salvat}
{Center for Exploration of Energy and Matter}{Bloomington, IN}{
\techlist{C++, Make, Git, Geant4, ROOT}
In my current position, I work on the development of my research group's simulation software.
Our detectors measure the elastic neutrino-argon scattering cross section as part of the COHERENT collaboration.
The simulation software I work on is written in C++ and makes use of Geant4, a semi-modern C++ library commonly used for particle physics simulations.
We also make use of the C++ library ROOT for data serialization and analysis.
This program is used by several researchers in the collaboration, and Git is used for version control.
Simulation runs are primarily performed on computing clusters to take advantage of parallel execution.
\\
I am primarily responsible for maintaining, modernizing, and adding new features to our simulation program.
One of the large projects I have worked on was to modernize how the detector geometry is specified.
I added support for GDML geometry specification, where an XML-like file containing the detector geometry is loaded at run-time.
Previously, this information was hard-coded into the simulation executable and any changes to the detector required recompilation.
I am currently working on updating the scintillation physics engine to support xenon doped liquid argon.
I am also implementing a regression testing system so that refactoring and modernization projects can be undertaken with greater confidence.
This project has given me great experience working on a moderate size code-base that was written by someone other than myself.
}%}}}

% Linac
\cventry{2019--2021}{Positron Converter Simulation and Modeling}%{{{
{Jim Shanks, David Sagan}
{Cornell Lab for Accelerator based Science and Education (CLASSE)}
{Ithaca, NY}{
\techlist{C++, CMake, Subversion, Geant4, GSL}
This project focused on developing simulation and modeling software for positron converters.
These particle accelerator components are used to produce positrons for use in electron-positron colliders.
I wrote this software from scratch using C++17.
The simulation component made heavy use of Geant4, and the modeling component used the GNU Scientific Library for data analysis and fitting.
I also wrote the user manual for this software, as well as a paper to be published in a peer reviewed journal.
This software is packaged as part of the Bmad library of accelerator simulation software, which uses a CMake-based build system and Subversion for version control.
}%}}}

% Tao GUI
\cventry{2019--2020}{Tao GUI Development}%{{{
{David Sagan}{CLASSE}{Ithaca, NY}{
\techlist{Python, Fortran}
Tao is Cornell's Tool for Accelerator Optics, a program used around the world for modeling particle accelerators written in Fortran 95.
I worked on developing a GUI for Tao using python.
We also developed a general purpose scripting interface for the program.
}%}}}

% Conservative learning
\cventry{2019--2020}{Conservative Machine Learning}%{{{
{Veit Elser}{Ithaca, NY}{}{
\techlist{C, Machine Learning}
My advisor and I explored an alternative machine learning algorithm referred to as conservative learning.
Unlike stochastic gradient descent, which is almost universally employed in today's machine learning algorithms, conservative learning aims to take the smallest step size possible when updating the weights of the neural network.
}%}}}

% PHA
\cventry{2017}{Potentially Hazardous Asteroid Interception}%{{{
{Louis Rubbo}{Conway, SC}{}{
\techlist{MATLAB}
During the summer of 2017, I worked at Coastal Carolina University on this personal project.
We analyzed interception techniques that could be employed to reach and deflect potentially hazardous asteroids.
Most of this work was done using MATLAB to model orbits and compute interception trajectories.
}%}}}

\subsection{Side Projects}

\cventry{2018--present}{Linux Tinkering}%{{{
{}{}{}{
\techlist{Linux, Shell scripting, GNU Core Utilities}
I have been using GNU/Linux as my primary operating system since 2018.
I am passionate about free software and try to use it in place of proprietary software whenever possible.
This hobby has made me very proficient at the command line.
I consider my self well versed in shell scripting and basic UNIX-like utilities, and I am very comfortable with Linux system administration tasks.
These are skills that add to my strengths as a software developer.
}%}}}

\cventry{2020--present}{Personal Website}%{{{
{}{}{}{
\techlist{Javascript, Node.js, HTML/CSS, MySQL, Linux}
I run my personal website, \url{johnmm.xyz}, on a Linux VPS hosted with Vultr.
I host a basic HTML and CSS site using NGINX, and have a personal email service running on the server as well.
This web-server also hosts a simple web app for chore tracking that I developed using client and server side JavaScript.
This service also uses MySQL for database management.
}%}}}

\subsection{Teaching}

% Three items {{{
\cventry{2020--2021}{Physics Associate Instructor (TA)}
{Indiana University Bloomington}{Bloomington, IN}{}{
As an associate instructor, I taught weekly discussion and lab sections for IU's general physics courses.
I also graded lab reports, quizzes, and exams, proctored exams, and held office hours.
}

\cventry{2017--2019}{MATH 1120 Course Assistant}
{Cornell University Math Department}{Ithaca, NY}{}{
As a course assistant, I was responsible for grading the homework from one to two sections of MATH 1120 each week.
I also held a study group session each week where students worked together on their homework and asked course assistants for help if necessary.
}

\cventry{2015--2016}{Private Tutor}{Self Employed}{Conway, SC}{}{
While I was high school senior, I worked as a private tutor for several students at Coastal Carolina University. 
Most of these students were taking introductory physics or calculus, though I did find some work tutoring upper-level physics courses as well.
This experience helped me learn a great deal about being an effective tutor and teacher.
}
%}}}

\subsection{Other}

\cventry{2016--2017}{Special Collections Processing Assistant}{Cornell University Division of Rare and Manuscript Collections}{Ithaca, NY}{}{
My primary role as a processing assistant was to convert existing paper and index card catalogs to Excel spreadsheets.  This occasionally required me to examine and label each item in a collection, many of which were from the 19th century or earlier.
\\
During my time as a processing assistant, I also developed a python script
for automatically converting between human readable date formats (e.g.
November 7, 1950) and machine readable date formats (e.g. 1950-11-07).
This allowed date conversion to be done quickly, whereas this task
previously needed to be done by hand.
}

\cventry{2011--2015}{Professional Musician}{}{Myrtle Beach, SC}{}{
I worked as a professional musician over the summers throughout middle and high school.
My family's band played hundreds of jobs each summer, both at resorts in Myrtle Beach and at private parties and events.
I also took private bass guitar lessons and was exposed to a wide variety of music genres.
}

\section{Computer skills}

\subsection{Programming Languages}%{{{

\cvlistdoubleitem{Modern C++}{C}
\cvlistdoubleitem{Python}{MATLAB}
\cvlistdoubleitem{JavaScript}{Shell scripting (bash, POSIX sh)}

\subsection{Markup Languages}

\cvlistdoubleitem{HTML/CSS}{LaTeX}

\subsection{Other Technologies}
\cvlistdoubleitem{Linux}{GDB}
\cvlistdoubleitem{Git}{CMake and Make}
\cvlistdoubleitem{Node.js}{SQL}%}}}


\section{References}
\begin{cvcolumns}%{{{
  \cvcolumn{Name}{\begin{itemize}
\item Rex Tayloe
\item Dan Salvat
\item David Sagan
\item Jim Shanks
\end{itemize}}
  \cvcolumn{Email}{
\emaillink{rtayloe@indiana.edu} \\
\emaillink{dsalvat@iu.edu} \\
\emaillink{david.sagan@cornell.edu} \\
\emaillink{shanks@cornell.edu} \\
}
  \cvcolumn{Relationship}{
Research Advisor \\
Research Advisor \\
Research Advisor \\
Research Advisor \\
}
\end{cvcolumns}%}}}


\clearpage

% recipient data
\newcommand{\company}{Rolls-Royce LibertyWorks}
\newcommand{\positionName}{Advanced Development Performance Specialist}
\recipient{\company}{13980 E CAPT WJ Nelson Dr\\
Odon, IN 47562}
\date{February 27, 2022}
\opening{Dear Sir or Madam,}
\closing{Sincerely,}

\makelettertitle

I am writing to apply for the \positionName{} position at \company{}.
I have several years of experience writing software in C++ and Python and a strong desire to learn new skills,
and I feel that I would be a great fit for the position.

As my resum\'{e} shows, my degrees are in physics and mathematics. However, I
do have several years of experience as a software developer through my
involvement with several research teams, and through my personal projects. I
feel that this professional experience has given me strong software engineering
and related skills that make me a great fit for your open position. In
addition, my background in physics and math has given me excellent analytical
and problem solving skills. These skills are essential in the software
development field, and I believe that my background sets me apart as a unique
applicant.

I am most proficient and comfortable writing C++. Most of my professional
software development experience has involved writing C++, and I am familiar
with a variety of C++ programming styles, both old and new. In addition, I have substantial experience with Python and Linux development.
I am also comfortable with many of the other technical skills you are looking for, such as algorithms, optimization, and AI/machine learning.

I hope that you will consider me for your team, and I look forward to
hearing from you. Thank you for your time.

\makeletterclosing



\end{document}
